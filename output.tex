\documentclass{beamer}
\usepackage[utf8]{inputenc}
\usepackage{natbib}
\usepackage{pgf-pie}
\usepackage{color}
\usepackage{pgfplots}
\usepackage{multicol}
\usepackage[T1]{fontenc}
\usepackage[utf8]{inputenc}
\usepackage[spanish]{babel}
\usepackage{amssymb, amsmath, amsbsy}
\usepackage{utopia} %font utopia imported
\usepackage{lmodern}
\usepackage{soul} 

\usetheme{Singapore}
\usecolortheme{default}

\definecolor{azure}{rgb}{0.0, 0.5, 1.0}
\definecolor{amethyst}{rgb}{0.6, 0.4, 0.8}
\definecolor{green}{rgb}{0.01, 0.75, 0.24}
\definecolor{pink}{rgb}{1.0, 0.65, 0.79}
\definecolor{green}{rgb}{0.01, 0.75, 0.24}
\definecolor{orange}{rgb}{0.93, 0.53, 0.18}

%------------------------------------------------------------
%This block of code defines the information to appear in the
%Title page





\title[] %optional
{An\'alisis de proceso de scanner para lenguage C}

\subtitle{Proyecto I: Compiladores e Intérpretes}

\institute{Tecnol\'ogico de Costa Rica \\
            Francisco Torres Rojas \\
}
\author{Carlos Adri\'an G\'omez Segura\\
        Kevin Jafet Zamora\\
        Gabriel Solórzano Chanto\\
    }
\date{ Abril 10 , 2019\\
        Semestre 1 }  % Se manda por parámetro



\logo{\includegraphics[height=0.75 cm]{/home/gabriel/Music/Lat/photos/tec.png}}

%End of title page configuration block
%------------------------------------------------------------



%------------------------------------------------------------
%The next block of commands puts the table of contents at the 
%beginning of each section and highlights the current section:

\AtBeginSection[]
{
  \begin{frame}
    \frametitle{Table of Contents}
    \tableofcontents[currentsection]
  \end{frame}
}
%------------------------------------------------------------
\makeatletter
\newcommand\Oricolor{%
  \let\set@color\beamerorig@set@color
  \let\reset@color\beamerorig@reset@color
  }
\makeatother 

\begin{document}

%The next statement creates the title page.
\frame{\titlepage}


%---------------------------------------------------------
%This block of code is for the table of contents after
%the title page
\begin{frame}
\frametitle{Table of Contents}
\begin{columns}

\column{0.5\textwidth}
\begin{center}
\tableofcontents    
\end{center}

\column{0.5\textwidth}
\begin{center}
    \includegraphics[height=5.5cm]{/home/gabriel/Music/Lat/photos/2.jpg}
\end{center}
\end{columns}
\end{frame}
%---------------------------------------------------------


\section{Proceso de scanning}

\begin{frame}{Descripción del proceso de de scanning.}
    

\begin{columns}

\column{0.5\textwidth}
El preproceso es el primer proceso de ejecución que se realiza en el momento de compilación.\\

Se encarga principalmente de:
\begin{itemize}
\item \#defines
\item \#includes
\end{itemize}

\column{0.5\textwidth}
\begin{center}
    \includegraphics[height=3.5cm]{/home/gabriel/Music/Lat/photos/1.jpg}
\end{center}

\end{columns}
\end{frame}


\begin{frame}
\frametitle{Cont. Descripción del proceso de de scanning.}


\begin{columns}
\column{0.5\textwidth}
La primera parte es el preproceso que se encarga de preparar el código para que luego el scanner lo utilice.
Se encarga de hacer lo includes y hacer el define\\ 





\column{0.5\textwidth}
Include, busca y carga código que se encuentra en la misma carpeta, en la que se está ejecutando el archivo, en el preproceso también se dirige al archivo, por ende, busca el código y lo agrega al archivo  deseado.\\
\end{columns}

\begin{center}
    \includegraphics[height=2.5cm]{/home/gabriel/Music/Lat/photos/4.jpg}
\end{center}

\end{frame}



\begin{frame}{Cont. Descripción del proceso de de scanning.}
\begin{columns}

\column{0.5\textwidth}
Define: busca y reemplaza en todo el código, lo establecido en la línea del Define, lo reemplaza por constantes.\\
Otra función:\\
\begin{itemize}
\item Eliminar los comentarios 
\end{itemize}

\column{0.5\textwidth}

\begin{center}
    \includegraphics[height=3 cm]{/home/gabriel/Music/Lat/photos/5.jpg}
\end{center}

\end{columns}
\end{frame}


%------------------------------------------------------------------------------
\begin{frame}{Cont. Descripción del proceso de de scanning.}

\begin{center}
Sección de scanner\\    
\end{center}



\begin{columns}

\column{0.5\textwidth}
El scanner se va a encargar de generar unidades léxicas mínimas  en este caso, tokens, a partir de Flex, recibe el programa C, y Flex reconocerá qué tipo de token es, el mismo va a llevar ciertos contadores sobre el programa para luego  ser procesadas en una presentación Beamer.

\column{0.5\textwidth}

\begin{center}
    \includegraphics[height=3 cm]{/home/gabriel/Music/Lat/photos/6.jpg}
\end{center}

\end{columns}
\end{frame}

%--------------------------------------------------------------------------------

\section{Herramienta Flex }

\begin{frame}
\frametitle{Descripción de la herramienta Flex.}
\begin{columns}

\column{0.5\textwidth}

\begin{center}
    \includegraphics[height=4.5cm]{/home/gabriel/Music/Lat/photos/3.jpg}
\end{center}

\column{0.5\textwidth}

Flex es una herramienta de gran ayuda que genera los analizadores lexicos.\\ 

A través de un conjunto de expresiones regulares, se busca coincidencias y ejecuta
instrucciones relacionadas a la expresiones.\\

\end{columns}

\end{frame}







\section{Análisis del programa }

\begin{frame}
\frametitle{Notación.}

\begin{center}
    \fcolorbox{azure}{azure}{\textit{Palabras Reservadas}}\\
    \fcolorbox{yellow}{yellow}{Carácteres Especiales}\\
    \fcolorbox{red}{red}{Operadores}\\
    \fcolorbox{orange}{orange}{\textit{Strings}}\\
    \fcolorbox{green}{green}{\textit{Identificadores}}\\
    \fcolorbox{amethyst}{amethyst}{\textbf{Constantes}}\\
    \fcolorbox{pink}{pink}{Errores Léxicos}
\end{center}

\end{frame}\begin{frame}{Análisis de código} 
\fcolorbox{pink}{pink}{\textbf{á}} \fcolorbox{pink}{pink}{\textbf{é}} \fcolorbox{pink}{pink}{\textbf{í}} \fcolorbox{pink}{pink}{\textbf{ó}} \fcolorbox{pink}{pink}{\textbf{ú}} \fcolorbox{pink}{pink}{\textbf{\$}} \fcolorbox{pink}{pink}{\textbf{\$}} \fcolorbox{pink}{pink}{\textbf{ä}} \fcolorbox{pink}{pink}{\textbf{ë}} \fcolorbox{pink}{pink}{\textbf{ö}}\\
\fcolorbox{pink}{pink}{\textbf{$\neg$}} \fcolorbox{pink}{pink}{\textbf{$\neg$}} \fcolorbox{pink}{pink}{\textbf{`}} \fcolorbox{pink}{pink}{\textbf{`}}\\
 \end{frame}
  \section{Histogramas del análisis}  \begin{frame} \frametitle{Histograma de la cantidad de tokens}  
 \begin{tikzpicture}  
 \begin{axis}[legend columns=3, 
   	x tick label style={ 
 /pgf/number format/1000 sep=}, 
 ylabel=Tokens,
 	enlargelimits=0.05, 
legend style={at={(0.5,-0.04)}, 
  	anchor=north,legend columns=-1},  
 	ybar interval=0.7,   
  ] 
  \addplot coordinates {(0,0) (5,5)}; 
 \addplot coordinates {(0,0) (5,5)}; 
 \addplot coordinates {(0,0) (5,5)}; 
 \addplot coordinates {(0,0) (5,5)}; 
 \addplot coordinates {(0,0) (5,5)}; 
 \addplot coordinates {(0,0) (5,5)}; 
 \addplot coordinates {(0,10) (5,5)}; 
  \legend{Palabras reservadas,Identificadores,Strings, Constantes, Símbolos Especiales, Operadores, Errores Léxicos } 
  \end{axis} 
 \end{tikzpicture} 
 \end{frame}  
  \begin{frame}  \frametitle{Gráfico circular del porcentaje de uso de tokens } 
  \begin{tikzpicture}[scale=0.9]  
    \pie[text=legend]{0.00 / Palabras reservadas: 0.00 \%, 0.00 / Identificadores: 0.00 \%, 0.00 / Strings: 0.00 \%, 0.00 / Constantes: 0.00 \%, 0.00 / Símbolos especiales: 0.00 \%, 0.00 / Operadores: 0.00 \%, 100.00 / Errores Léxicos: 100.00 \%}  
  \end{tikzpicture} 
 \end{frame}  
  \section{Referencias }\begin{frame}{Referencias}Levine, J. (2009). \textit{Flex \& Bison: Text Processing Tools.} O'Reilly Media, Inc.\end{frame}\end{document}